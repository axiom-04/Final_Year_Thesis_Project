% !TEX root = holder_file.tex
\chapter{Introduction}
\noindent In today's world, derivatives are frequently used in the financial markets. A financial instrument known as a derivative is one whose value is based on the cost of an underlying variable. Risk hedging, speculation, and arbitrage are all possible uses of derivatives. Futures contracts, forward contracts, options, and swaps are a few examples of common derivatives. Here, we'll talk about possibilities, including American and European ones and will suggest efficient methods for each in this project. \\[2mm]
\noindent A key development in contemporary finance is \textbf {Option Pricing}. In the case of option pricing, a tons of factors work behind this mechanism such as stock price, strike price, market volatility, interest rates etc. Here, the gap between the underlying security's current stock price and strike price defines the payoff. Moreover, the Market
volatility plays a vital role on option prices because it displays the potential risk
of sudden price changes within the market and the rate of interest is the return
on risk-free investments.\\[-3mm]

\noindent Black and Scholes established a European option valuation model. (1973). The development of the Black-Scholes model made substantial use of Taylor's expansion. A contract known as a European option lets the The option holder may later exercise the right to purchase or sell stocks. Unlike European options, a subset of the more common American options, permit the Anytime up until the expiration date, the holder may exercise the options. \\[2mm]
Either the option is European or American, Solving a practical problem is always a difficult task. Because practical problem means there are tons of uncertainties. So, then Numerical approaches are necessary for pricing choices in situations when analytical solutions are either not available or difficult to calculate. \\[2mm]
The objective of using Monte Carlo Method in the project is to address the pricing issue for American and European alternatives. Monte Carlo method is a computational simulations that rely on repeated random sampling to obtain numerical results. In this project, we have used risk neutral methodology in MC simulation. Using this method we evaluated the price for European Call, American Call and Barrier Put option. Then we have showed some variance reduction method for reducing the computation time and increasing the accuracy of these prices. We basically discussed in details about antithetic method for variance reduction. Afterwards, some comparison are shown with respect to Black Scholes Merton Model (BSM) in the final chapter. 
Our project is disscussed as follows:

\begin{itemize}
	\item In Chapter 2, we have disscussed about some preliminary concepts such as
	derivatives, different types of markets, contracts, traders, different types of
	options, payoffs, some basic definitions, stochastic process, Black-Scholes Model and etc.
	\item In Chapter 3, we have discussed about the history of Monte Carlo Method, concept, formula and algorithm. Then the discussion took a turn towards option pricing using MC method, variance reduction process for European Call option. Moreover, in this project some analysis has been shown on American Call option and Barrier Put option.
	\item In Chapter 4, this chapter is created focusing on only the comparison of each options.
	\item In Chapter 5, we make conclusion by comparing the outcomes from different
	methods discussed in this project.
\end{itemize}