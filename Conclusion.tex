% !TEX root = holder_file.tex
\chapter{Conclusion}
	\noindent In this project, we have discussed various topics related to derivatives, in particular options, namely, European, American and Barrier. We have used Monte Carlo Simulation Method for pricing these options. \\[2mm]
	
	\noindent We have shown MC method for European Call option. In table 4.1 we can see that the absolute difference is very low of BSM and MC. So, Monte Carlo Method is an excellent method for evaluating option prices. But this method is computationally inefficient. If we increase the number of iteration, it would take a long time to compute the result. So, for reducing the computational cost and saving time, we have discussed some variance reduction method specially Antithetic Method. After using this, the approximation become better. 
	\noindent Moreover, we have shown same output for American Call Option and Barrier Put Option. In both cases, Monte Carlo Method is showing results which are almost matching to BSM Method. 
	\noindent Here, we’ve implemented these procedures discussed in this
	project into Python coding for numerical results as well as graphical representation. We also have a fantastic experience throughout our project
	using the most scientific typesetting  \LaTeX . Finally, we like to conclude that this
	work will be a good guideline for further study of pricing other options like exotic
	options, asian options etc.